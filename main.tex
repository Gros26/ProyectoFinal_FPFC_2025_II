\documentclass{article}
\usepackage{graphicx} % Required for inserting images

\title{Proyecto}
\author{Sebastian Ovi}
\date{November 2025}

\begin{document}

\maketitle

\section*{Correctitud de la función \texttt{itinerariosTiempo}}

Sea \texttt{itinerariosTiempo(vuelos,aeropuertos)} una función que retorna otra
función $G$ tal que, dados dos códigos de aeropuerto $c_1$ y $c_2$,
$G(c_1,c_2)$ devuelve la lista de itinerarios mínimos en tiempo total
de viaje entre dichos aeropuertos. Internamente, la función utiliza un
procedimiento recursivo \texttt{buscarTiempo} que explora todas las
rutas posibles, descarta aquellas con ciclos y finalmente selecciona
las de menor tiempo total.

Para demostrar la correctitud de esta implementación, consideremos la
siguiente especificación matemática del problema:

\begin{itemize}
    \item Sea $V$ el conjunto de todos los vuelos disponibles.
    \item Un itinerario desde $c_1$ hasta $c_2$ es una secuencia de vuelos
          $\langle v_1,\dots,v_m \rangle$ tal que:
          \begin{enumerate}
              \item $v_1.\mathrm{Org} = c_1$,
              \item $v_m.\mathrm{Dst} = c_2$,
              \item $v_i.\mathrm{Dst} = v_{i+1}.\mathrm{Org}$ para todo $i$,
              \item No existen ciclos: ningún aeropuerto aparece dos veces.
          \end{enumerate}
    \item El tiempo total del itinerario es la suma de las duraciones de cada vuelo.
    \item La función debe devolver exactamente los tres itinerarios de menor tiempo total,
          o menos si existen menos rutas posibles.
\end{itemize}

Denotemos por $F(c)$ el conjunto de todos los itinerarios válidos desde un
aeropuerto $c$ hasta el destino $d$ (fijo para cada llamada).
Sea \texttt{buscarTiempo(actual, destino, visitados)} la implementación recursiva.
Queremos demostrar que la función computa exactamente $F(c)$.

\subsection*{Definición de Correctitud}

Diremos que la implementación es correcta si, para todo aeropuerto
$A$ y destino $D$:

\[
\texttt{buscarTiempo}(A, D, \{A\}) = F(A)
\]

y la función \texttt{itinerariosTiempo} selecciona efectivamente los
itinerarios de mínimo tiempo.

\subsection*{Demostración por inducción estructural}

La función es naturalmente recursiva sobre la estructura del grafo
dirigido de vuelos, por lo que utilizamos inducción sobre la \emph{longitud}
de los itinerarios posibles desde un aeropuerto dado.

Sea $T(a)$ el conjunto de itinerarios válidos desde el aeropuerto $a$
hasta el destino $d$. Demostraremos que:

\[
\texttt{buscarTiempo}(a,d,visitados) = T(a)
\]

para todo aeropuerto $a$ y conjunto de visitados con $a \in visitados$.

\subsubsection*{Caso base}

Si $a = d$, la especificación indica que existe exactamente un itinerario:
la secuencia vacía que representa ``no hay más vuelos que tomar''.

La implementación retorna:

\[
\texttt{List(Nil)}
\]

lo cual coincide con la especificación:

\[
T(d) = \{\langle \;\rangle\}.
\]

Por tanto, el caso base es correcto.

\subsubsection*{Paso inductivo}

Supongamos que para todos los aeropuertos $x$ tales que toda ruta
posible desde $x$ al destino tiene longitud estrictamente menor que
las rutas desde $a$, la implementación satisface:

\[
\texttt{buscarTiempo}(x, d, visitados \cup \{x\}) = T(x).
\]

Demostremos la correctitud para $a \neq d$.

La implementación construye el conjunto:

\[
V(a) = \{ v \in V \mid v.\mathrm{Org} = a \;\wedge\; v.\mathrm{Dst} \notin visitados \}.
\]

Es decir, todos los vuelos posibles desde $a$ sin entrar en ciclo.  
Esto coincide exactamente con la definición matemática del conjunto de vuelos válidos.

Para cada vuelo $v \in V(a)$, la implementación calcula recursivamente:

\[
\texttt{buscarTiempo}(v.\mathrm{Dst}, d, visitados \cup \{v.\mathrm{Dst}\}).
\]

Por hipótesis inductiva, esto retorna exactamente el conjunto de todos
los itinerarios desde $v.\mathrm{Dst}$ hasta $d$.

La implementación construye entonces todos los itinerarios desde $a$ como:

\[
\bigcup_{v \in V(a)} \{v\} \mathbin{::} T(v.\mathrm{Dst}),
\]

que coincide \emph{exactamente} con la definición matemática de $T(a)$.

Dado que:

\begin{itemize}
    \item no se omite ningún vuelo posible,
    \item no se incluyen vuelos que generen ciclos,
    \item todas las rutas se generan mediante concatenación correcta,
\end{itemize}

queda demostrado que:

\[
\texttt{buscarTiempo}(a,d,visitados) = T(a).
\]

\subsection*{Mínimo tiempo total}

Una vez generado $T(c_1)$, la función \texttt{itinerariosTiempo}:

\begin{enumerate}
    \item Calcula la duración total de cada itinerario.
    \item Ordena los itinerarios por su tiempo total.
    \item Toma los tres primeros mediante \texttt{take(3)}.
\end{enumerate}

Esto corresponde exactamente a seleccionar:

\[
\text{los 3 elementos mínimos del conjunto } T(c_1)
\]

bajo el orden total definido por el tiempo total de viaje.

Dado que la función genera el conjunto completo de itinerarios válidos
y luego selecciona los de tiempo mínimo, concluimos que la implementación
es correcta.

\subsection*{Conclusión}

Por inducción sobre la profundidad de búsqueda en el grafo y por análisis
funcional de la minimización, queda demostrado que:

\[
\texttt{itinerariosTiempo(vuelos,aeropuertos)(c_1,c_2)}
\]

\emph{devuelve exactamente los tres itinerarios de menor tiempo total de viaje},
cumpliendo con la especificación formal del problema.


\end{document}
